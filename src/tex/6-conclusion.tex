\chapter{Conclusion}

In this thesis we examined the novel cloud computing paradigm of serverless computing particularly from the point of view of web application development. Going back to the four original research questions, we were first inquiring into the motivations behind serverless migration. This question was answered in form a literature review in Chapter \ref{cha:serverless}. We first traced the origins of the serverless paradigm from utility computing on to containers and microservices. We then attempted to define the paradigm along with its two distinct manifestations of BaaS and FaaS, paying special attention to how they differ from earlier cloud computing models. The literature review also delved into serverless use cases, providers, security issues and economics, finishing with an in-depth look into the paradigm's drawbacks and limitations. In summary the paradigm's main advantages are reduced operational overhead, configuration-free elasticity and a pricing model based on actual utilization instead of reservation.

The next two research questions concerned serverless design patterns. In Chapter \ref{cha:patterns} we first surveyed existing serverless patterns as well as adapted patterns from other relevant computing areas. Then in Chapter \ref{cha:migration} we applied the patterns in migrating a web application to serverless architecture, attempting to identify gaps in the patterns. The outcome of this process -- and also the main design artifact of this thesis -- are the five new serverless design patterns introduced in Section \ref{sec:newPatterns}.

The final research question dealt with how serverless migration affects applications quality-wise. This question was addressed in Chapter \ref{cha:evaluation} from developmental, performance and economic perspectives. First in a qualitative assessment serverless  architecture was found to benefit from easier modularization and rapid deployment. Conversely pain points were identified in local development and testing as well as in monitoring due to the high level of distribution. Second, the two implementations' performance and scaling characteristics were compared through a stress test: here the serverless architecture was found to live up to its purported elasticity with response times staying constant regardless of traffic rate. Finally evaluating the migration's economic implications we arrived at the same conclusion as previous surveyors \parencite[including][]{baldini17currentTrends}: the serverless cost benefit is most noticeable in case of bursty and inconsistent workloads whereas steady and constant utilization can be cheaper to host in other paradigms.

The main limitation of this work is the shortage of practical experience working with the introduced design patterns. The migrated application represents a singular, rather narrow use case and cannot thus account for all possible corner cases one might encounter while designing a serverless application. An interesting opportunity for future research would be to crowdsource further patterns by surveying experienced developers on their serverless usage; this is also the avenue taken in the yet unpublished work on serverless anti-patterns by \textcite{taibi19antipatterns}. Another potential area of future research involves thoroughly benchmarking the cost-optimizing patterns such as Local Threader (\ref{subsec:LocalThreads}) and Prefetcher (\ref{subsec:prefetcher}) to identify the cases where the reduction in hosting costs outweighs the implementation cost. The next step from that would be to implement software to automatically identify such cost-optimizing opportunities in existing serverless systems.
