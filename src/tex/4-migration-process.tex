\chapter{Migration process}

``Do the existing patterns have gaps or missing parts, and if so, can we come up with improvements or alternative solutions?''

``The fourth chapter describes the process of migrating an existing web application to serverless architecture. The patterns discovered in the previous chapter are utilized to implemented various typical web application features on a serverless platform. In cases where existing patterns prove insufficient or unsuitable as per the target application's characteristics, modifications or new patterns are proposed.''

Implement a subset of the target app in a serverless style, utilizing the surveyed patterns and keeping log of the tricky parts. In case the patterns prove unsuitable for the given problem, try to come up with an alternative solution.

Describe the web application to migrate.

Decide on the parts to migrate. Should demonstrate the features and limitations of serverless as outlined above. Possible features include
\begin{itemize}
  \item A REST API endpoint to showcase API Gateway and synchronous invocation. Shouldn't require any big changes to application code.
  \item Interaction between multiple services to demonstrate distributed transactions.
  \item A scheduled (cron) event.
  \item Interacting with an external SaaS service like Twilio, Auth0. Demonstrate event-driven invocation.
  \item others?
\end{itemize}

\section{Literature on serverless migration}[sec:migrationLiterature]

TODO Go through earlier papers on serverless migration.

\textcite{lloyd18migration}: `` we present on a case study migration of the Precipitation Runoff Modeling System (PRMS), a Java-based environmental modeling application to the AWS Lambda serverless platform. We investigate performance and cost implications of memory reservation size, and evaluate scaling performance for increasing concurrent workloads. We then investigate the use of Keep-Alive workloads to preserve serverless infrastructure to minimize cold starts and ensure fast performance after idle periods for up to 100 concurrent client requests. We show how Keep-Alive workloads can be generated using cloud-based scheduled event triggers, enabling minimization of costs, to provide VM-like performance for applications hosted on serverless platforms for a fraction of the cost.''

Two case studies by \textcite{adzic2017serverless} already mentioned in section \ref{sec:economics}.
