\chapter{Serverless design patterns} \label{cha:patterns}

In this chapter we take a look at serverless design patterns. Design patterns describe commonly accepted, reusable solutions to recurring problems \parencite{hohpe2004enterprise}. A design pattern is not a one-size-fits-all solution directly translatable into software code, but rather a formalized best practice that presents a common problem in its context along a general arrangement of elements that solves it \parencite{gamma94designPatterns}. The patterns in this chapter are sourced from scientific literature on serverless computing as well as cloud provider documentation. Also literature on object-oriented patterns \parencite{gamma94designPatterns}, SOA patterns \parencite{rotem12soa} and enterprise integration patterns \parencite{hohpe2004enterprise} was surveyed for applicable practices.

Enterprise integration patterns, as serverless is all about integrations. \textcite{hohpe2004enterprise} present a number of asynchronous messaging architectures in the seminal book on EIP. While predating the whole serverless phenomenon the patterns are still relevant. Hohpe even demonstrated implementing one of his patterns on top of Google's serverless platform \href{http://www.enterpriseintegrationpatterns.com/ramblings/google_cloud_functions.html}{in a blog post}. E.g. patterns like Idempotent Receiver, Dead-letter Channel as well as the 4 more general integration styles of File Transfer, Shared Database, RPC and Messaging. Many patterns implemented internally by FaaS platforms already!

SOA patterns: FaaS functions are self-contained nanoservices these might have some relevance. SOA patterns \parencite{rotem12soa} include Saga, Decoupled Invocation and others. As with EIP, some patterns are already implemented by the FaaS platform.

FaaSification: \textcite{spillner17transformpython} describes an automated approach to transform monolithic Python code into modular FaaS units by partially automated decomposition. Doesn't really seem suitable for the web application migration process covered in this thesis but worth mentioning.

\section{Orchestration patterns} \label{sec:orchestrationPatterns}

Function orchestration and composition.

\subsection{Routing function} \label{subsubsec:routingFunction}

FaaS version of the GoF Component pattern.

\subsection{Function chain} \label{subsubsec:functionChain}

Chain functions to avoid timeouts.

\subsection{Periodic invocation} \label{subsubsec:periodicInvocation}

Schedule function invocations using cron-like systems.

\subsection{State machine} \label{subsubsec:stateMachine}

AWS Step Functions etc.

\subsection{Fat Client} \label{subsubsec:fatClient}

Let client orchestrate workflows.


\section{API/Integration patterns} \label{sec:apiPatterns}

Integrating with external systems.

\subsection{API composition} \label{subsubsec:apiComposition}

Hide multiple API calls under a single function.

\subsection{API aggregation} \label{subsubsec:apiAggregation}

Hide a sequential multi-step API call under a single function.

\subsection{API async} \label{subsubsec:apiAsync}

Turn a synchronized API into an async one.

\subsection{Legacy API Proxy/Staged migration} \label{subsubsec:legacyApi}

Replace a legacy API with a new one step by step, a.k.a. Strangler.

\subsection{Separate FaaS handler from core logic} \label
{subsubsec:separateHandler}

Separate FaaS handler core logic in code level.

\section{Data management/access patterns} \label{sec:dataManagementPatterns}

Managing state and accessing external resources.

\subsection{Externalized State} \label{subsubsec:externalizedState}

Store function state in external storage.

\subsection{Valet Key} \label{subsubsec:valetKey}

Sign tokens for clients to directly access resources.

\subsection{Least privilege IAM role} \label{subsubsec:LeastprivilegeIAMrole}

Minimize attack surface by reducing function access roles to bare minimum.

\section{Event patterns} \label{sec:eventPatterns}

Asynchronous messaging/event patterns.

\subsection{Event processing} \label{subsubsec:Eventprocessing}

Trigger a function as a result of event occurrence.

\subsection{Fan-out events} \label{subsubsec:FanoutEvents}

Trigger multiple actions from a single event.

\subsection{Fan-out/fan-in} \label{subsubsec:FanoutFanin}

Split event handling into parallel functions.

\subsection{Pipes and filters} \label{subsubsec:PipesAndFilters}

Handle event stream with a pipeline of small functions.

\section{Performance and scalability patterns} \label{sec:perfPatterns}

Address FaaS performance issues.

\subsection{Function warming} \label{subsubsec:FunctionWarming}

Ping a function intermittently to avoid cold starts.

\subsection{Oversized function} \label{subsubsec:OversizedFunction}

Choose maximum memory allocation to access faster CPU resources and improve cold start latency.

\subsection{Singleton} \label{subsubsec:Singleton}

Take advantage of function execution context to avoid reinitializing function dependencies.

\section{Resiliency and availability patterns} \label{sec:resiliencyPatterns}

Maximize serverless system resiliency.

\subsection{Bulkhead} \label{subsubsec:Bulkhead}

Isolate high-latency code into separate functions to avoid resource contention.

\subsection{Flow control/throttling} \label{subsubsec:Flow control/throttling}

Throttle invocations to avoid DDoSing yourself.

\subsection{Circuit breaker} \label{subsubsec:Circuit breaker}

Keep track of component availability to avoid cascading failures.

