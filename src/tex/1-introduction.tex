\chapter{Introduction} \label{cha:introduction}

Cloud computing has in the past decade emerged as a veritable backbone of modern economy, driving innovation both in industry and academia as well as enabling scalable global enterprise applications. Just as adoption of cloud computing continues to increase, the technologies in which the paradigm is based on have continued to progress. Recently the development of novel virtualization techniques has lead to the introduction of \textit{serverless computing}, a novel form of cloud computing based on ephemeral resources that scale up and down automatically and are billed for actual usage at a millisecond granularity. The main drivers behind serverless computing are reduced operational costs through more efficient cloud resource utilization as well as improved developer productivity achieved by shifting provisioning, load balancing and other infrastructure concerns to the service provider. \parencite{buyya2017manifesto}

As an appealing economic proposition, serverless computing has attracted significant interest in the industry. This is illustrated for example by its appearance in the 2017 Gartner Hype Technologies Report \parencite{walker17gartnerHype}. By now most of the prominent cloud service providers have introduced their own serverless platforms, promising capabilities that make writing scalable web services easier and cheaper \parencite[e.g.][]{awslambda0218, google18cloudFunctions, ibm18cloudFunctions, microsoft18azureFunctions}. A number of high-profile use cases have been presented in the literature \parencite{cncf18serverlessWG}, and some researchers have gone as far as to predict that ``serverless computing will become the default computing paradigm of the Cloud Era, largely replacing serverful computing and thereby bringing closure to the Client-Server Era'' \parencite{jonas19berkeleyView}. \textcite{baldini17currentTrends} however note a lack of corresponding degree of interest in academia despite a wide variety of technologically challenging and intellectually deep problems in the space.

One of the open problems identified in literature concerns the discovery of serverless design patterns: how do we compose the granular building blocks of serverless into larger systems? \parencite{baldini17currentTrends} \textcite{varghese18next} contend that one challenge hindering the widespread adoption of serverless will be the radical shift in the properties that a programmer will need to focus on, from latency, scalability and elasticity to those relating to the modularity of an application. Considering this it is unclear to what extent our current patterns apply and what kind of new patterns are best suited to optimize for the paradigm's unique characteristics and limitations. The object of this thesis is to fill the gap by re-evaluating existing design patterns in the serverless context and proposing new ones through an exploratory migration process.

\section{Research problem}

The research problem addressed by this thesis distills down to the following four questions:
\begin{enumerate}
	\item Why should a web application be migrated to serverless?
	\item What kind of patterns are there for building serverless web applications?
	\item Do the existing patterns have gaps or missing parts, and if so, can we come up with improvements or alternative solutions?
	\item How does migrating a web application to serverless affect its quality?
\end{enumerate}

The first two questions are addressed in the theoretical part of the thesis. Question 1 concerns the motivation behind the thesis and introduces serverless migration as an important and relevant business problem. Question 2 is answered by surveying existing literature for serverless patterns as well as other, more general patterns thought suitable for the target class of applications.

The latter questions form the constructive part of the thesis. Question 3 concerns the application and evaluation of surveyed patterns. The surveyed design patterns are used to implement a subset of an existing conventional web application in a serverless architecture. In case the patterns prove unsuitable for any given problem, alternative solutions or extensions are proposed. The last question consists of comparing the migrated portions of the app to the original version and evaluating whether the posited benefits of serverless architecture are in fact realized.

\section{Outline}

The thesis is structured as follows: the second chapter serves as an introduction to the concept of serverless computing. The chapter describes the main benefits and drawbacks of the platform, as well as touching upon its internal mechanisms and briefly comparing the main service providers. Extra emphasis is placed on how the platform's limitations should be taken into account when designing web applications.

The third chapter consists of a survey into existing serverless design patterns and recommendations. Applicability of other cloud computing, distributed computing and enterprise integration patterns is also evaluated.

The fourth chapter describes the process of migrating an existing web application to serverless architecture. The patterns discovered in the previous chapter are utilized to implemented various typical web application features on a serverless platform. In cases where existing patterns prove insufficient or unsuitable as per the target application's characteristics, modifications or new patterns are proposed.

The outcome of the migration process is evaluated in the fifth chapter. The potential benefits and drawbacks of the serverless platform outlined in chapter 2 are used to reflect on the final artifact. The chapter includes approximations on measurable attributes such as hosting costs and performance as well as discussion on the more subjective attributes like maintainability and testability. The overall ease of development -- or developer experience -- is also assessed since it is one of the commonly reported pain points of serverless computing \parencite{van2017spec}.

The final chapter of the thesis aims to draw conclusions on the migration process and the resulting artifacts. The chapter contains a summary of the research outcomes and ends with recommendations for further research topics.
