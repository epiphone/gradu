\documentclass[utf8,english]{gradu3}
% If you are writing a Bachelor's Thesis, use the following instead:
%\documentclass[utf8,bachelor,english]{gradu3}

\usepackage{graphicx} % for including pictures

\usepackage{amsmath} % useful for math (optional)

\usepackage{booktabs} % good for beautiful tables

% NOTE: This must be the last \usepackage in the whole document!
\usepackage[bookmarksopen,bookmarksnumbered,linktocpage]{hyperref}

\addbibresource{bibliography.bib} % The file name of your bibliography database

\begin{document}

\title{Migrating a web application to serverless architecture}
\translatedtitle{Web-sovelluksen siirtäminen serverless-arkkitehtuuriin}
\studyline{Master's Thesis in Information Technology}
\avainsanat{%
  serverless,
  FaaS,
  arkkitehtuuri,
  pilvilaskenta,
  web-sovellukset}
\keywords{
  serverless,
  FaaS,
  architecture,
  cloud computing,
  web applications}
\tiivistelma{%
  Tämä kirjoitelma on esimerkki siitä, kuinka
  {gradu3}-tutkielmapohjaa käytetään.  Se sisältää myös
  käyttöohjeet ja tutkielman rakennetta koskevia ohjeita.

  Tutkielman tiivistelmä on tyypillisesti lyhyt esitys, jossa
  kerrotaan tutkielman taustoista, tavoitteesta, tutkimusmenetelmistä,
  saavutetuista tuloksista, tulosten tulkinnasta ja johtopäätöksistä.
  Tiivistelmän tulee olla niin lyhyt, että se, englanninkielinen
  abstrakti ja muut metatiedot mahtuvat kaikki samalle sivulle.

  Sen tulee kertoa täsmälleen samat asiat kuin englannikielinen
  abstrakti.
}
\abstract{%
  This document is a sample {gradu3} thesis document class
  document.  It also functions as a user manual and supplies
  guidelines for structuring a thesis document.

  The abstact is typically short and discusses the background, the
  aims, the research methods, the obtained results, the interpretation
  of the results and the conculsions of the thesis.  It should be so
  short that it, the Finnish translation, and all other meta
  information fit on the same page.

  The Finnish tiivistelmä of a thesis should usually say exactly the same
  things as the abstract.
}

\author{Aleksi Pekkala}
\contactinformation{\texttt{alvianpe@student.jyu.fi}}
% use a separate \author command for each author, if there is more than one
\supervisor{Oleksiy Khriyenko}
% use a separate \supervisor command for each supervisor, if there
% is more than one

\maketitle

% \begin{thetermlist}
% \item[FaaS] Function as a Service.
% % \item[\LaTeX] A system, built on top of \TeX\
% %   \parencite{knuth86:_texbook}, for typesetting structured
% %   documents \parencite[see][]{lamport94:_latex}.  Its current version
% %   is \LaTeXe.
% \end{thetermlist}

\mainmatter

\chapter{Introduction}

\begin{quote}
It is a good idea to start the Introduction with the main thesis statement or research question of the thesis.  After that, it is a good idea to clarify things by defining any necessary terms. Definitions after the thesis statement!  Also, don't babble in the introduction. The introduction is also a good place to discuss why your thesis statement is scientifically or practically relevant and interesting.  Ideally, it would be relevant and interesting from both the scientific and the practical point of view. It would also be excellent if you explained, in the introduction, what your contribution is; that is, what such knowledge your thesis contains that you have investigated personally instead of reading it from somewhere. The contribution could well be, that you have personally checked the truth of a claim you found in a book or article.  At the end of the Introduction, it is customary to briefly explain the structure of the thesis -- what each chapter is about.
\end{quote}

In a nutshell: serverless computing is a relatively new form of cloud computing that enables efficient resource utilization and elasticity, thus bringing about potential savings in hosting costs and increases in developer productivity \parencite{robert2016serverlessarchitectures}. As a rather new development there's a lack of research in serverless computing. Multiple authors, e.g. \textcite{baldini17currentTrends}, \textcite{fox17} and \textcite{van2017spec} note the identification of FaaS-specific patterns (composing functions into larger systems) as an important research topic. The object of this thesis is to fill that gap by a) surveying existing patterns in literature and b) evaluate, extend and propose new patterns by migrating parts of a web application into serverless architecture.

\section{Research questions}

Introduce the following research questions:

1. Why should a web app be migrated to FaaS?

2. What kind of patterns are there for building serverless web application backends?

3. Do the existing patterns have gaps or missing parts, and if so, can we come up with improvements or alternative solutions?

4. How does migrating a web app to FaaS effect its quality?

\section{Outline}

The thesis is structured as follows: the second chapter serves as an introduction to the concept of serverless computing. The chapter describes the main benefits and drawbacks of the platform, as well as touching upon its internal mechanisms and briefly comparing the main service providers. Extra emphasis is placed on how the platform's limitations should be taken into account when designing web application backends.

The third chapter consists of a survey into existing serverless design patterns. The applicability of other cloud computing, distributed computing and enterprise integration patterns is also evaluated in the chapter.

The fourth chapter describes the process of migrating an existing web application to serverless architecture. The patterns discovered in the previous chapter are utilized to implemented various typical web application features on a serverless platform. In cases where existing patterns prove insufficient or unsuitable as per the target application's characteristics, modifications or new patterns are proposed. This chapter forms the exploratory or constructive part of the thesis.

The outcome of the migration process is evaluated in the fifth chapter. The potential benefits and drawbacks of the serverless platform outlined in chapter 2 are used to reflect on the final artifact. The chapter includes approximations on measurable attributes such as hosting costs and performance as well as discussion on the more subjective attributes like maintainability and developer experience.

The final chapter of the thesis aims to draw conclusions on the migration process and the resulting artifacts. The chapter contains a summary of the research outcomes and ends with recommendations for further research topics.

\chapter{Serverless computing}

\begin{quote}
The goal of the theoretical part of a thesis is to develop the theoretical background required in the thesis.  The idea is that a reader of the thesis should, based on just the thesis itself, be able to understand all the special concepts and methods used in the thesis. A good thesis also gives well-argued reasons for why exactly these concepts and methods are in use in the thesis (with the main alternatives given in the literature mentioned).
\end{quote}

\section{What is serverless?}

Definition and implications of serverless computing. Main features or tenets. Benefits, use cases, notable users? \textcite{robert2016serverlessarchitectures} has a thorough description of the platform.

Two historical/evolutionary views: on-premises infra -> grid -> IaaS -> PaaS -> serverless and on-prem -> VMs -> containers -> serverless.

How does serverless relate to concepts such as FaaS, SOA, cloud-native, microservices, FaaS, containers, ...?

\section{Service providers}

\textcite{lynn2017preliminary} provide an overview and multi-level feature analysis of seven enterprise serverless computing platforms.

\section{Execution models and triggers}

Describe the two supported execution models, synchronous and asynchronous, and how they relate to application design. The former is used to build a typical request-response flow, e.g. a REST API endpoint, whereas the latter relates to pub-sub and other event-driven flows. Give examples on the kind of triggers supported by serverless platforms (HTTP calls, messaging, database events, ...).

\section{Drawbacks and limitations}

What to take into consideration when migrating to serverless?

- cold start \parencite{lloydserverless}

- the need for circuit breakers (risk of DDoSing yourself) when interacting with non-serverless components like a database. Mention novel serverless database services like Google's Cloud Spanner and AWS Aurora.

- SLAs

- vendor lock-in

\textcite{baldini17trilemma} identify three competing constraints: functions should be considered as black boxes; function composition should obey a substitution principle with respect to synchronous invocation; and invocations should not be double-billed.

\textcite{robert2016serverlessarchitectures} and \textcite{adzic2017serverless} respectively list a number of limitations.

\chapter{Serverless design patterns}

Survey of serverless design patterns. How to compose individual functions into larger systems?

\textcite{sbarski2017serverless} introduce the following five patterns: Command, Messaging Priority queue, Fan-out and Pipes and filters.

Enterprise integration patterns by \textcite{hohpe2004enterprise}.

Serverless programming model embraces modularization and isolation.

\textcite{adzic2017serverless} suggest embracing the platform by 3 typical aspects: use distributed authorization, let clients orchestrate workflows and allow clients to directly connect to AWS resources.

\chapter{Migration process}

Decide on parts to migrate.

A simple REST API to showcase API Gateway and synchronous invokation, shouldn't require any big changes to application code.

Interacting with an external SaaS service like Twilio, Auth0. Demonstrate event-driven invocation.

Something to do with timed events, cron.

An analytics pipeline.

Transactions, eventual consistency. Demonstrate double-billing.

Possibly split into another evaluation chapter? Evaluating costs, performance, etc.

\chapter{Evaluation}

Evaluation the outcome of the migration process. Performance, costs, ...?

\chapter{Conclusion}

\begin{quote}
The last chapter of a thesis is the Conclusion (some authors use
Conculsions, instead).  Keep it short, and discuss what one can
conclude about the thesis statement or research question given in the
Introduction, in light of all that has been written in the thesis.
The Conclusion is also the place to discuss any limitations and
weaknesses of the thesis (especially those that cast doubt on the
reliabliity of the results given in the thesis), if they have not been
already discussed, for example in a Discussion chapter.  It is also
customary to state, what further research might be beneficial in light
of this thesis.

If the Conclusion threatens to become too long, it is a good idea to
split the interpretation of the results into its own chapter, often
called Discussion, making Conclusion short and sweet.
\end{quote}

\printbibliography

\end{document}
