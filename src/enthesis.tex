\documentclass[utf8,english]{gradu3}
% If you are writing a Bachelor's Thesis, use the following instead:
%\documentclass[utf8,bachelor,english]{gradu3}

\usepackage{graphicx} % for including pictures

\usepackage{amsmath} % useful for math (optional)

\usepackage{booktabs} % good for beautiful tables

% NOTE: This must be the last \usepackage in the whole document!
\usepackage[bookmarksopen,bookmarksnumbered,linktocpage]{hyperref}

\addbibresource{bibliography.bib} % The file name of your bibliography database

\begin{document}

\title{Migrating a web application to serverless architecture}
\translatedtitle{Web-sovelluksen siirtäminen serverless-arkkitehtuuriin}
\studyline{Master's Thesis in Information Technology}
\avainsanat{%
  serverless,
  FaaS,
  arkkitehtuuri,
  pilviarkkitehtuuri,
  web-sovellukset}
\keywords{
  serverless,
  FaaS,
  architecture,
  cloud architecture,
  web applications}
\tiivistelma{%
  Tämä kirjoitelma on esimerkki siitä, kuinka
  {gradu3}-tutkielmapohjaa käytetään.  Se sisältää myös
  käyttöohjeet ja tutkielman rakennetta koskevia ohjeita.

  Tutkielman tiivistelmä on tyypillisesti lyhyt esitys, jossa
  kerrotaan tutkielman taustoista, tavoitteesta, tutkimusmenetelmistä,
  saavutetuista tuloksista, tulosten tulkinnasta ja johtopäätöksistä.
  Tiivistelmän tulee olla niin lyhyt, että se, englanninkielinen
  abstrakti ja muut metatiedot mahtuvat kaikki samalle sivulle.

  Sen tulee kertoa täsmälleen samat asiat kuin englannikielinen
  abstrakti.
}
\abstract{%
  This document is a sample {gradu3} thesis document class
  document.  It also functions as a user manual and supplies
  guidelines for structuring a thesis document.

  The abstact is typically short and discusses the background, the
  aims, the research methods, the obtained results, the interpretation
  of the results and the conculsions of the thesis.  It should be so
  short that it, the Finnish translation, and all other meta
  information fit on the same page.

  The Finnish tiivistelmä of a thesis should usually say exactly the same
  things as the abstract.
}

\author{Aleksi Pekkala}
\contactinformation{\texttt{alvianpe@student.jyu.fi}}
% use a separate \author command for each author, if there is more than one
\supervisor{Oleksiy Khriyenko}
% use a separate \supervisor command for each supervisor, if there
% is more than one

 % you don't need this line in a thesis

\maketitle

% \preface
% This is where you can write a preface for your thesis.  Most theses
% don't have prefaces, but if you write one, keep it short (at least one
% page).

% The preface should discuss more the thesis process than the content of
% the thesis.  For example, if there is something out of the ordinary in
% your choice of a thesis topic or if something out of the ordinary
% happened during its prepararion, the preface is where you could write
% about it.  It is also customary in a preface to thank by name those
% persons who helped you with your thesis -- at least your supervisor,
% your spouse and your children, if any.  (Your family likely will have
% helped you by encouraging and supporting you.)

% The preface is typically in the first person (``I'').  It is also common
% to sign it.

% Jyväskylä, \today

% \bigskip

% The Author

\begin{thetermlist}
\item[FaaS] Function-as-a-Service.
% \item[\LaTeX] A system, built on top of \TeX\
%   \parencite{knuth86:_texbook}, for typesetting structured
%   documents \parencite[see][]{lamport94:_latex}.  Its current version
%   is \LaTeXe.
\end{thetermlist}

\mainmatter

\chapter{Introduction}

Introos \parencite{adzic2017serverless}
cloud computing is important
trending towards PaaS, containerization etc
serverless is one of these trends
serverless has key benefits, economically attractive
serverless requires changes in application architecture
lack of research on patterns

It is a good idea to start the Introduction with the main thesis
statement or research question of the thesis.  After that, it is a
good idea to clarify things by defining any necessary
terms.\footnote{Definitions after the thesis statement!  Also, don't
  babble in the introduction.}  The introduction is also a good place
to discuss why your thesis statement is scientifically or practically
relevant and interesting.  Ideally, it would be relevant and
interesting from both the scientific and the practical point of view.
It would also be excellent if you explained, in the introduction, what
your contribution is; that is, what such knowledge your thesis
contains that you have investigated personally instead of reading it
from somewhere.  The contribution could well be, that you have
personally checked the truth of a claim you found in a book or
article.  At the end of the Introduction, it is customary to briefly
explain the structure of the thesis -- what each chapter is about.

\chapter{Serverless}

Description and implications of serverless computing. Main features or tenets.
Relation to SOA, cloud-native, microservices, FaaS.
What to take into consideration when migrating to serverless?

- cold start

- bottlenecks and circuit breakers when interacting with non-serverless components like a database. Spanner and Aurora?

- vendor lock-in

\textcite{baldini17trilemma} identify three competing constraints: functions should be considered as black boxes; function composition should obey a substitution principle with respect to synchronous invocation; and invocations should not be double-billed.

\textcite{lynn2017preliminary} provide an overview and multi-level feature analysis of seven enterprise serverless computing platforms.

\textcite{adzic2017serverless} list a number of limitations.

The goal of the theoretical part of a thesis is to develop the
theoretical background required in the thesis.  The idea is that a
reader of the thesis should, based on just the thesis itself, be able
to understand all the special concepts and methods used in the thesis.
A good thesis also gives well-argued reasons for why exactly these
concepts and methods are in use in the thesis (with the main
alternatives given in the literature mentioned).

The best way to present and use the theoretical background depends on
what the thesis is like.  The theoretical part of a
mathematico-theoretical work differs considerably fron the theoretical
part of a constructive software development work; quite different from
both is the theoretical part of a quantitative or qualitative
empirical study that is based on the traditions of the behavioral or
the social sciences.  Reading other theses of the same type, as well
as similar published research reports, will give you a good impression
of what is required of your own thesis.

\chapter{Serverless design patterns}

Survey of serverless design patterns. How to compose individual functions into larger systems?

Will there be patterns for building serverless solutions? How do we combine low granularity basic building blocks of serverless into bigger solutions? How are we going to decompose apps into functions so that they optimize resource usage? For example how do we identify CPU-bound parts of applications built to run in serverless services? Can we use well-defined patterns for composing functions and external APIs? What should be done on the server vs. client (e.g., are thicker clients more appropriate here)? Are there lessons learned that can be applied from OOP design patterns, Enterprise Integration Patterns, etc.? \parencite{baldini17currentTrends}

\textcite{sbarski2017serverless} introduce the following five patterns: Command, Messaging Priority queue, Fan-out and Pipes and filters.

Include something from Enterprise Integration Patterns?

\textcite{adzic2017serverless} suggest connecting clients directly to services without any need for controllers in the middle. Embrace the platform.




\section{After the theory}

The theoretical part is followed by your contribution:
\begin{itemize}
\item In a mathematico-theoretical thesis it is usually a sequence of
  definitions and lemmas of your own devising, which then culminate in
  the proof of your main theorem.
\item In a constructive thesis it is usually a computer program or
  other artefact that you have made yourself.
\item In an empirical thesis it is a set of empirical results obtained
  by applying a empirical research method.
\end{itemize}

You should present your contribution with precision, giving reasons
for the choices you have made.  You should follow the best practices
of the research tradition you are using.

\chapter{Using the literature}

The theoretical part is almost always based solely on the literature.
When discussing your contribution, you may also need to cite the
literature.

Remember to avoid plagiarism.  If you copy, either verbatim or with
slight changes (or, example, in your own translation) text from some
source, make it clear to the reader.  Mark your quotes (using
quotation marks or some other clear manner) and give a precise
citation.  If you do not quote verbatim, mark any changes you have
made.  In most situations, however, it is better to use your own
words, based on more than one source.  Even then, give clear
citations.



\chapter{Conclusion}

The last chapter of a thesis is the Conclusion (some authors use
Conculsions, instead).  Keep it short, and discuss what one can
conclude about the thesis statement or research question given in the
Introduction, in light of all that has been written in the thesis.
The Conclusion is also the place to discuss any limitations and
weaknesses of the thesis (especially those that cast doubt on the
reliabliity of the results given in the thesis), if they have not been
already discussed, for example in a Discussion chapter.  It is also
customary to state, what further research might be beneficial in light
of this thesis.

If the Conclusion threatens to become too long, it is a good idea to
split the interpretation of the results into its own chapter, often
called Discussion, making Conclusion short and sweet.

After Conclusion, there is the bibliography, indicated by the
\string\printbibliography\ command, followed by appendices, if any.

\printbibliography

\appendix
\section{Moving from gradu2 to gradu3}

Moving an incomplete thesis from gradu2 to gradu4 is not particularly
difficult.  The first thing to do is to change gradu2 into gradu3 in
the \string\documentclass\ command.  Most of the options given to it
must be removed, as they are not supported.  A ``kandi'' option is
changed into ``bachelor''; any ``english'' option is retained, and so
is ``utf8'', ``latin1'', or ``latin9''.

\end{document}
